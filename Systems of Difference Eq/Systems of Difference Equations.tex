\documentclass[10pt,usenames,dvipsnames]{beamer}
\usetheme{CambridgeUS}
%\usetheme{Boadilla}
\definecolor{myred}{RGB}{163,0,0}
%\usecolortheme[named=`blue]{structure}
\usecolortheme{dove}
\usefonttheme[]{professionalfonts}
\usepackage[english]{babel}
\usepackage{amsmath,amsfonts,amssymb}
\usepackage{xcolor}
\usepackage{tikz}
\usepackage{pgfplots}
\pgfplotsset{compat=newest,compat/show suggested version=false}
\usetikzlibrary{arrows,shapes,calc,backgrounds}
\usepackage{bm}
\usepackage{textcomp}
\usepackage{gensymb}
\usepackage{verbatim}
\usepackage{paratype}
\usepackage{mathpazo}
\usepackage{listings}
\usepackage{csvsimple}

\newcommand{\cov}{\mathsf{cov}}
\newcommand{\corr}{\mathsf{corr}}
\newcommand{\var}{\mathsf{Var}}
\newcommand{\plim}{\mathrm{plim\ }}
\newcommand{\E}{\mathsf{E}}
\newcommand{\Est}{\mathsf{Est.Var}}
\newcommand{\Asy}{\mathsf{Asy.Var}}
\newcommand{\Esta}{\mathsf{Est.Asy.Var}}
\newcommand{\tr}{\mathrm{tr}}
\newcommand{\Prob}{\mathrm{Prob}}
\newcommand{\Med}{\mathsf{Med}}
\newcommand{\rank}{\mathsf{rank}}
\newcommand{\argmin}{\mathsf{arg\,min}}

\newcommand{\cc}[1]{\texttt{\textcolor{blue}{#1}}}

\definecolor{ttcolor}{RGB}{0,0,1}%{RGB}{163,0,0}

% Number theorem environments
\setbeamertemplate{theorem}[ams style]
\setbeamertemplate{theorems}[numbered]

% Reset theorem-like environments so that each is numbered separately
%\usepackage{etoolbox}
%\undef{\definition}
\theoremstyle{plain}
\newtheorem{thm}{Theorem}
\theoremstyle{definition}
\newtheorem{defn}{\translate{Definition}}

\definecolor{ttcolor}{RGB}{0,0,1}%{RGB}{163,0,0}
\definecolor{mygray}{RGB}{248,249,250}

% Change colours for theorem-like environments
\definecolor{mygreen1}{RGB}{0,96,0}
\definecolor{mygreen2}{RGB}{229,239,229}
\setbeamercolor{block title}{fg=white,bg=mygreen1}
\setbeamercolor{block body}{fg=black,bg=mygreen2}

\lstdefinestyle{numbers}{numbers=left, stepnumber=1, numberstyle=\tiny, numbersep=10pt}
\lstdefinestyle{MyFrame}{backgroundcolor=\color{yellow},frame=shadowbox}

\lstdefinestyle{rstyle}%
{language=R,
	basicstyle=\footnotesize\ttfamily,
	backgroundcolor = \color{mygray},
	commentstyle=\slshape\color{green!50!black},
	keywordstyle=\bfseries\color{blue!50!black},
	identifierstyle=\color{blue},
	stringstyle=\color{orange},
	%escapechar=\#,
	rulecolor = \color{mygray}, 
	showstringspaces = false,
	showtabs = false,
	tabsize = 2,
	emphstyle=\color{red},
	frame = single} 

\setbeamertemplate{navigation symbols}{}

\lstset{language=R,frame=single}

\AtBeginSection{\frame{\sectionpage}}

% Remove Section 1, Section 2, etc. as titles in section pages
\defbeamertemplate{section page}{mine}[1][]{%
	\begin{centering}
		{\usebeamerfont{section name}\usebeamercolor[fg]{section name}#1}
		\vskip1em\par
		\begin{beamercolorbox}[sep=12pt,center]{part title}
			\usebeamerfont{section title}\insertsection\par
		\end{beamercolorbox}
	\end{centering}
} 

\setbeamertemplate{section page}[mine]  

\hypersetup{colorlinks, urlcolor=blue, linkcolor = myred} 

\title{R406: Applied Economic Modelling with Python}
\subtitle{\textcolor{myred}{Systems of Difference Equations}}
\author[Kaloyan Ganev,  Andrey Vassilev]{Kaloyan Ganev (main author) \\
	Andrey Vassilev (minor modifications)}

\date{} 

\begin{document}
\maketitle

\begin{frame}[fragile]
\frametitle{Lecture Contents}
\tableofcontents
\end{frame}

\section{Introduction}
\begin{frame}[fragile]
\frametitle{Introduction}
\begin{itemize}
	\item So far we considered single-variable equations
	\item Also, we stuck to the autonomous case; we shall do the same in the current topic
	\item Suppose that instead of having just one equation in one variable, we have the following set of two equations in two variables:
	\[
		\begin{array}{lcl}
			y_{t} = ax_{t-1} + b y_{t-1}\\
			x_{t} = cx_{t-1} + d y_{t-1}
		\end{array}
	\]
	\item Apparently, this is already a system of difference equations
	\item Such systems turn out to be very useful in describing relationships among economic variables which are not confined to a single direction of causality
\end{itemize}
\end{frame}

\begin{frame}[fragile]
\frametitle{A note on classification}
\begin{itemize}
	\item \textbf{Autonomous} vs. \textbf{non-autonomous} systems were already introduced on-the-fly
	\item \textbf{Non-homogeneous} systems are those in which at least one equation is non-homogeneous
	\item Likewise, \textbf{non-linear} systems of equations are the ones in which there is at least one non-linear equation
	\item As with the single-equation case, we will consider only linear systems
\end{itemize}
\end{frame}

\section{Analysis of linear systems of equations}
\begin{frame}[fragile]
\frametitle{Matrix form of systems}
\begin{itemize}
	\item The simple homogeneous system example given earlier can be written as follows:
	\[
		\left[\begin{matrix}
			x_{t}\\
			y_{t}
		\end{matrix}\right] = 
		\left[\begin{matrix}
			a & b \\
			c & d
		\end{matrix}\right]
		\left[\begin{matrix}
			x_{t-1}\\
			y_{t-1}
		\end{matrix}\right] \quad (*)
	\]
	\item Using matrix shorthand, this is the same as:
	\[
		\mathbf{u}_{t} = \mathbf{Au}_{t-1}
	\]
	where $\mathbf{u}_{t} = [x_{t},y_{t}]'$, and $\mathbf{A}$ is the coefficient matrix
	\item By analogy, the non-homogeneous counterpart to this system would look as follows:
	\[
		\mathbf{u}_{t} = \mathbf{Au}_{t-1} + \mathbf{b}
	\]
	where $\mathbf{b}$ is a vector of conformable size (can be time-dependent but let's forget about this for the time being)
\end{itemize}
\end{frame}

\begin{frame}[fragile]
\frametitle{Equilibrium}
\begin{itemize}
	\item Take $(*)$ are the reference point
	\item Equilibrium implies that $x_{t} = x^{*}$ and $y_{t} = y^{*},\, \forall t$
	\item Therefore, in equilibrium:
	\[
		\left[\begin{matrix}
			x^{*}\\
			y^{*}
		\end{matrix}\right] = 
		\left[\begin{matrix}
			a & b \\
			c & d
		\end{matrix}\right]
		\left[\begin{matrix}
			x^{*}\\
			y^{*}
		\end{matrix}\right],
	\]
	or:
	\[
		\mathbf{u}^{*} = \mathbf{Au}^{*}
	\]
	\item Move everything to the LHS of the equation:
	\[
		\mathbf{u}^{*} - \mathbf{Au}^{*} = \mathbf{0} \Leftrightarrow ( \mathbf{I-A})\mathbf{u}^{*} = \mathbf{0}
	\]
\end{itemize}
\end{frame}

\begin{frame}[fragile]
\frametitle{Equilibrium (2)}
\begin{itemize}
	\item If $\mathbf{I-A}$ is a non-singular matrix, the equilibrium corresponds to  $\mathbf{u}^{*} = \mathbf{0}$
	\item Now, return to the non-homogeneous system; equilibrium again requires that $\mathbf{u}_{t} = \mathbf{u}^{*},\, \forall t$, i. e.:
	\[
		\mathbf{u}^{*} = \mathbf{Au}^{*} + \mathbf{b}
	\]
	\item Move everything except $\mathbf{b}$ to the LHS:
	\[
		\mathbf{u}^{*} - \mathbf{Au}^{*} = \mathbf{b}
	\]
\end{itemize}
\end{frame}

\begin{frame}[fragile]
\frametitle{Equilibrium (3)}
\begin{itemize}
	\item This is the same as:
	\[
		(\mathbf{I-A})\mathbf{u}^{*} = \mathbf{b}
	\]
	\item If $\mathbf{I-A}$ is a non-singular matrix, then the equilibrium of the system is found where:
	\[
		\mathbf{u}^{*} = (\mathbf{I-A})^{-1}\mathbf{b}
	\]
	\item This looks very much like the result for the single-equation case which shouldn't be surprising
\end{itemize}
\end{frame}

\begin{frame}[fragile]
\frametitle{Stability}
\begin{itemize}
	\item As in the single-equation case, in order to analyse stability, we need to solve the system
	\item The procedure that is followed is analogical
	\item Start with the homogeneous case:
	\[
		\mathbf{u}_{t} = \mathbf{Au}_{t-1} = \mathbf{A}(\mathbf{Au}_{t-2}) = \mathbf{A}^{2}\mathbf{u}_{t-2} = \ldots = \mathbf{A}^{t}\mathbf{u}_{0}
	\]
	\item In the non-homogeneous case the solution is:
	\[
		\mathbf{u}_{t} = \mathbf{A}^{t}\mathbf{u}_{0} + (\mathbf{I + A} + \mathbf{A}^{2} + \ldots + \mathbf{A}^{t-1})\mathbf{b}
	\]
\end{itemize}
\end{frame}

\begin{frame}[fragile]
\frametitle{Stability (2)}
\begin{itemize}
	\item Note that a non-homogeneous system can be reduced to a homogeneous one by expressing it in deviations from equilibrium
	\item Therefore, after subtracting $\mathbf{u}^{*}$ from both sides of the matrix equation, we have:
	\[
		\mathbf{u}_{t} - \mathbf{u}^{*}  = \mathbf{A}(\mathbf{u}_{t-1} - \mathbf{u}^{*})
	\]
	\item Denote $\mathbf{z}_{t} = \mathbf{u}_{t} - \mathbf{u}^{*}$; then:
	\[
		\mathbf{z}_{t}  = \mathbf{A}\mathbf{z}_{t-1}
	\]
	\item This is a first-order homogeneous system
	\item Therefore, from now on the discussion can proceed with homogeneous systems without losing generality with respect to the non-homogeneous case
\end{itemize}
\end{frame}

\begin{frame}[fragile]
\frametitle{Stability (3)}
\begin{thm}
	If the matrix $ \mathbf{A} $ has two distinct eigenvalues $r$ and $s$, then there exists a matrix $ \mathbf{V} = [\mathbf{v}^{r} \quad \mathbf{v}^{s}]$ composed of the eigenvectors corresponding to $r$ and $s$ such that:
	\[
		\mathbf{D} = 
		\left[\begin{matrix}
			r & 0\\
			0 & s
		\end{matrix}\right] = 
		\mathbf{V}^{-1}\mathbf{AV}
	\]
\end{thm}
\begin{itemize}
	\item If we pre-multiply both sides of the Theorem's result by $\mathbf{V}$ and post-multiply them by $\mathbf{V}^{-1}$, we get:
	\[
		\mathbf{V}\mathbf{D}\mathbf{V}^{-1} = \mathbf{V}\mathbf{V}^{-1}\mathbf{AV}\mathbf{V}^{-1} = \mathbf{A}
	\]
	\item It is also easy to see that $\mathbf{A}^{t} = \mathbf{V}\mathbf{D}^{t}\mathbf{V}^{-1}$
\end{itemize}
\end{frame}

\begin{frame}[fragile]
\frametitle{Stability (3)}
\begin{itemize}
	\item Using this, we can write:
	\[
		\mathbf{u}_{t} = \mathbf{A}^{t}\mathbf{u}_{0} = \mathbf{V}\mathbf{D}^{t}\mathbf{V}^{-1}\mathbf{u}_{0}
	\]
	\item But this can be stated explicitly as follows:
	\[
		\mathbf{u}_{t} = \mathbf{V}
		\left[\begin{matrix}
			r^{t} & 0\\
			0 & s^{t}
		\end{matrix}\right]		
		\mathbf{V}^{-1}\mathbf{u}_{0}
	\]
\end{itemize}
\end{frame}

\begin{frame}[fragile]
\frametitle{Theorems}
\begin{thm}
	A necessary and sufficient condition for the system of difference equations to be globally asymptotically stable is that all eigenvalues of the matrix $\mathbf{A}$ have moduli strictly less than 1.
\end{thm}

\begin{thm}
	If all eigenvalues of $\mathbf{A}$ have moduli strictly less than 1, the difference equation is globally asymptotically stable and every solution converges to the constant vector $\mathbf{(I - A)}^{-1}\mathbf{b}$.
\end{thm}
\end{frame}


\begin{frame}
	\frametitle{Transformation to autonomous form}
	\framesubtitle{}
		\begin{itemize}\itemsep1em
		\item Consider the non-autonomous equation \[ y_{t+1} = f(t, y_{t}), \quad t = 0, 1, 2, \ldots \]
		\item We can introduce a variable $ x_t $ and consider the modified system 
			\[
		\begin{array}{l}
			y_{t+1} = f(x_t, y_t)\\
			x_{t+1} = x_t + 1
		\end{array}
		\] together with the initial condition $ x_0=0 $
		\item The resulting system is autonomous, which has been achieved at the expense of an increase in the dimension of the original problem
		\item Sometimes such a transformation can facilitate the analysis of the system 
	\end{itemize}
\end{frame}

\begin{frame}[fragile]
	\frametitle{Order reduction}
	\framesubtitle{}
		\begin{itemize}
		\item Consider the equation \begin{equation}
			y_{t+p} = f(y_{t}, y_{t+1}, \ldots, y_{t+p-1}), \quad t = 0, 1, 2, \ldots
			\label{eq:pthorder}
		\end{equation} 
		\item Introduce the following variables
		\[
		\begin{array}{lll}
			x_{1,t} & = & y_t\\
			x_{2,t} & = & y_{t+1}\\
			x_{3,t} & = & y_{t+2}\\
			& \cdots &\\
			x_{p,t} & = & y_{t+p-1}
		\end{array}
		\]
		\item Then we can write the system
		\begin{equation}
					\begin{array}{lll}
				x_{1,t+1} & = & x_{2,t}\\
				x_{2,t+1} & = & x_{3,t}\\
				& \cdots &\\
				x_{p-1,t+1} & = & x_{p,t}\\
				x_{p,t+1} & = & f(x_{1,t},x_{2,t},\ldots,x_{p,t})
			\end{array}
			\label{eq:pdimsys}
		\end{equation}
	\end{itemize}
\end{frame}

\begin{frame}[fragile]
	\frametitle{Order reduction (2)}
	\framesubtitle{}
	\begin{itemize}
		\item This approach trades the $p$th-order equation \eqref{eq:pthorder} for the first-order system of $p$ equations \eqref{eq:pdimsys}
		\item The representation of a problem as a system of equations can be more convenient to work with in certain cases
	\end{itemize}
\end{frame}


\end{document}